\section{Tamanho das coalizões no parlamentarismo}

O que um partido ganha entrando numa coalizão? E quais fatores explicam por que algumas coalizões possuem mais partidos do que outras? A literatura sobre o parlamentarismo, mais antiga e extensa, nos fornece um bom ponto de partida para encontrar algumas dessas respostas. Grosso modo, três são as principais variáveis explicativas apontadas por ela: \textit{distribuição de cadeiras}, \textit{preferências ideológicas} e \textit{instituições}. As instituições porque estruturariam o processo de negociação e montagem de governos e, assim, delimitariam o número de coalizões que poderiam ser efetivamente formadas (STR\O{}M et al., \citeyear{strom1994}); já o número de cadeiras e a posição ideológica dos partidos, porque indicariam quais dessas coalizões seriam mais viáveis (AXELROD, \citeyear{axelrod1970}).

O marco inicial dessa literatura é bastante conhecido. Políticos procurariam usufruir de cargos públicos, que são bens escassos e de uso exclusivo, e, para tanto, formariam uma maioria na assembléia para controlar o governo. Dentre todas as coalizões que os permitiriam fazer isso, a ótima seria aquela que contaria com o mínimo de membros, a \textit{minimum winning coaltion} (MWC) (RIKER, (\citeyear{riker1962}). Na perspectiva dos políticos, isso seria vantajoso porque minimizaria o número de pessoas com quem eles teriam que dividir cargos, que são fixos no curto-prazo. Como todos enfrentam a mesma situação, a cooperação prospera e coalizões mínimas, mas capazes de satisfazer o critério de decisão majoritária, surgiriam. Embora não tenha sido o primeiro a formular esse modelo\footnote{Antes dele, Gamsom (\citeyear{gamson1961}) o formalizou de forma ampla; ele próprio, contudo, afirma que seu modelos inspira-se numa tradição alguns anos mais antiga, oriunda das primeiras contribuições à teoria dos jogos.}, Riker (\citeyear{riker1962}) o aplicou à situação concreta de formação de maiorias numa assembleia e derivou daí uma proposição bastante precisa sobre a coalizão que surgiria dada um distribuição de cadeiras\footnote{Cabe notar, do que fica implícito nesta proposição, que basta apenas a defecção de um membro da coalizão para que esta perca o controle do \textit{spoil}. Na prática, contudo, uma coalizão alternativa deve substituir a anterior para que este \textit{spoil} mude de mãos. Isto ocorre porque, na maioria dos países parlamentaristas, no caso de ausência de uma maioria alternativa, um governo \textit{caretaker} assume, geralmente composto por membros do governo anterior e comprometida em não alterar o \textit{status quo} (LAVER e SCHEPSLE, \citeyear{laver1996}, p. 47-8)}.

A simplicidade desse modelo \textit{office-seeking} \footnote{Para uma discussão sobre outros modelos \textit{office-based}, ver Martin e Stevenson (\citeyear{martin2001}) e Crombez (\citeyear{crombez1996}).}, entretanto, foi justamente o principal objeto de crítica da literatura subsequente. Entre outros, Luebbert (\citeyear{luebbert1986}) e Str\o{}m (\citeyear{strom1990}) mostraram que coalizões minoritárias não apenas surgiam com frequência, mas também que estas eram tão estáveis e poderiam durar por tanto tempo quanto suas correlatas majoritárias. Deste modo, Riker criou um \textit{puzzle} enorme: por que a oposição permite a sobrevivência de gabinetes minoritários? Ou bem existiria alguma compensação (\textit{side-payment}) aos partidos de fora do governo, ou então a utilidade destes seria derivada de outros locais.
 
Na esteira desse problema, emergiu a ideia de que não apenas cargos importavam, mas também a ideologia. A não ocupação de cargos poderia ser compensada, ou mesmo substituída, pela implementação de uma agenda legislativa da preferência dos partidos. Uma primeira explicação para isso seria a de que os membros de uma coalizão, por exemplo, buscariam diminuir os conflitos entre eles. Como nem sempre numa barganha dois partidos conseguem firmar acordo - seja por desconfiança mútua, falta de informações sobre as consequências do acordo, entre outros -, dois partidos quaisquer em coalizão poderiam incluir nela todos aqueles que estivessem situados entre eles numa escala ideológica unidimensional para amenizar esse problema, o que poderia gerar coalizões maiores do que as \textit{MWC} (AXELROD, 1970)\footnote{A explicação de Axelrod (1970) é a seguinte: se o potencial de conflito entre dois partidos aumenta quanto maior for a distância entre eles numa dimensão qualquer, e se o potencial de conflito numa coalizão é a média dos potenciais de conflitos entre todos os pares não repetidos de partidos, coalizões com menor variação nas preferências dos seus membros, isto é, coalizões mais homogêneas, gerariam maiores ganhos de estabilidade.}. Com isso, a manutenção da coalizão se tornaria mais fácil porque o novo integrante assumiria, de forma prática, o papel de mediano da coalizão, evitando que este tivesse que ser negociado. No caso das coalizões mínimas, a explicação poderia ser ainda mais direta: um partido de oposição pode simplesmente se beneficiar das políticas do governo e, por isso, não ter incentivos para derrubá-lo. Entre outros, a principal vantagem de fazer isso seria a de se poupar de desgastes com a administração pública, particularmente quando medidas impopulares precisam ser implementadas. Antecipando esse comportamento, o partido encarregado de formar uma coalizão, o \textit{formateur}, pode estrategicamente manter partidos ideologicamente próximos fora do governo e reter inteiramente o executivo. Tacitamente, contudo, uma coalizão existiria (STR\O{}M, \citeyear{strom1990}).

Distribuição de cadeiras e de preferências, portanto, podem influenciar o tamanho de uma coalizão, mas não fazem isso num vácuo institucional. Um conjunto de regras estrutura o processo de formação de gabinetes no parlamentarismo, determinando as estratégias de cada partido. Enquanto que, em tese, uma assembléia fragmentada oferece diversas possibilidades de coalizão, certas regras podem alterar drasticamente esse número (STR\O{}M et al., \citeyear{strom1994}). Por exemplo, dependendo de como o partido \textit{formateur} é escolhido, outros partidos podem recusar propostas de integrar a coalizão -- desde que tenham chances de se tornar o próximo \textit{formateur} caso as negociações em curso falhem, o que justamente lhes dá incentivos para não cooperar (AUSTEN-SMITH e BANKS, \citeyear{austen1988}). A divisão do governo em jurisdições específicas comandadas por ministérios, por sua vez, também pode dificultar a acomodação de muitos partidos num gabinete, pois torna difícil o monitoramento de cada ministério individualmente e aumenta as chances de que surjam conflitos intragovernamentais (LAVER e SHEPSLE, \citeyear{laver1996}). Deste modo, a consideração das regras que delimitam os cursos de ação disponíveis aos partidos seriam necessárias para tornar mais precisos os modelos sobre formação de gabinetes multipartidários.

Em suma, de acordo com essa literatura discutida, o contexto institucional e o número e as preferências dos partidos explicariam boa parte da variação no tamanho das coalizões formadas no parlamentarismo. O número de cadeiras seria um bom indicador da necessidade do partido \textit{formateur} buscar ou não uma coalizão e do número de partidos em coalizão necessários para se obter maioria. Preferências ideológicas, por outro lado, apontariam para como os partidos governariam e, portanto, seriam bons indicadores da probabilidade de que dois partidos quaisquer viessem a cooperar. Dependendo da posição e do número de cadeiras dos demais, coalizões mínimas ou sobredimensionadas podem surgir. Por fim, as regras que estruturam o processo de formação de coalizões e exercício do governo também deveriam ser levadas em conta para lidar com variações entre países e contextos, bem como para entender como a sequência de movimentos e as opções disponíveis a cada partido levam a equilíbrios diferentes dos que modelos \textit{office} ou \textit{policy-seeking} isoladamente indicariam. 
