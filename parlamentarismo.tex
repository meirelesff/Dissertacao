\subsection{Governos de coalizão no parlamentarismo}

O que um partido ganha entrando numa coalizão? Quais regras estruturam esse processo? Quem controla a agenda de negociação? Alguém pode vetar determinada coalizão de surgir? A literatura sobre o parlamentarismo, mais antiga e extensa, nos fornece um bom ponto de partida para encontrar algumas dessas respostas. Em especial, \textit{distribuição de cadeiras}, \textit{preferências} e \textit{instituições} são apontados por esta como as principais variáveis que podem influir no tamanho das coalizões governamentais. Esquematicamente, o fio condutor nessa literatura passa de (1) pressupor que políticos maximizam \textit{office} e cooperam para atingir esse fim para, (2),  pressupor que políticos maximizam \textit{policy} e barganham para atingir esse fim; com o \textit{institutionalist turn}, por sua vez, (3) algumas instituições passam a ser consideradas para resolver problemas de ciclicidade nas preferências e explicar certas variações no tamanho e heterogeneidade de coalizões. Grosso modo, há aí uma mini-história da ciência política moderna\footnote{Uma reconstituição fiel desse debate, incluindo estudos que não se encaixam nesse esquema, não poderia ser feita aqui. Para revisões mais exaustivas dessa literatura, ver Laver e Schofield (\citeyear{laver1998}), Martin e Stevenson (\citeyear{martin2001}) e CITAR OUTRO MAIS RECENTE.}.

O argumento dos primeiros modelos é bastante conhecido. Políticos procurariam entrar no governo para usufruir de cargos, que são bens escassos e de uso exclusivo; para fazer isso, contudo, precisam obter maioria absoluta no congresso para controlar o governo; dentre todos os grupos possíveis de serem formados para conseguir isso, o ótimo é aquele que ultrapassa a maioria com o mínimo de membros, a \textit{minimum winning coaltion} (MWC). Embora não tenha sido o primeiro a formular esse modelo\footnote{Antes dele, Gamsom (\citeyear{gamson1961}) o formalizou de forma ampla; ele próprio, contudo, afirma que seu modelos inspira-se numa tradição alguns anos mais antiga, oriunda das primeiras contribuições à teoria dos jogos feita por von Neumann CITAR ALGO DELE.}, Riker (\citeyear{riker1962}) o aplicou à situação concreta de formação de maiorias numa assembleia e derivou daí uma proposição bastante precisa sobre a coalizão surgiria, dada um distribuição de cadeiras: aquela que controla o governo com o mínimo de membros\footnote{Cabe notar, do que fica implícito nesta proposição, que basta apenas a defecção de um membro da coalizão para que esta perca o controle do \textit{spoil}. Na prática, contudo, uma coalizão alternativa deve substituir a anterior para que este \textit{spoil} mude de mãos. Isto ocorre porque, na maioria dos países parlamentaristas, no caso de ausência de uma maioria alternativa, um governo \textit{caretaker} assume, geralmente composto por membros do governo anterior e comprometida em não alterar o \textit{status quo} (LAVER e SCHEPSLE, \citeyear{laver1996}, p. 47-8)}. Na perspectiva dos políticos, isso seria vantajoso porque minimizaria o número de pessoas com quem eles teriam que dividir cargos, que são fixos no curto-prazo. Como todos enfrentam a mesma situação, a cooperação prospera. Deste modo, coalizões mínimas, mas capazes de satisfazer o critério de decisão majoritária, surgiriam.

A precisão e a economia de parâmetros desse modelo e variações\footnote{Para uma discussão sobre outros modelos \textit{office-based}, ver Martin e Stevenson (\citeyear{martin2001}) e Crombez (\citeyear{crombez1996}).} são, contudo, o principal motivo pelo qual ele foi criticado pela literatura subsequente. Estudos empíricos posteriores acumularam resultados o contrariando. Por exemplo, Luebbert (\citeyear{luebbert1986}) e Strom (\citeyear{strom1990}) não só mostraram que coalizões minoritárias surgiam com frequência, mas também que estas eram tão estáveis e poderiam durar por tanto tempo quanto suas correlatas majoritários. No fim das contas, Riker criou um \textit{puzzle} enorme: como é que a oposição permite a sobrevivência de gabinetes minoritários? Ou bem existiria alguma compensação (\textit{side-payment}) aos partidos fora do governo, ou então a utilidade dos partidos estaria sendo derivada de outro local.

O que a literatura posterior mostrou é que coalizões de tamanhos diferentes podiam ser explicadas com base nessa última possibilidade. Cargos podem ser compensados, ou mesmo substituídos, pela implementação de uma agenda legislativa da preferência dos partidos. Uma primeira explicação disso seria a de que partidos numa coalizão buscariam diminuir os conflitos por políticas entre eles. Para Axelrod (\citeyear{axelrod1970}), como nem sempre em situações de barganha o ponto médio entre dois partidos pode ser atingido - seja por incerteza, assimetria de informações, entre outros -, dois partidos quaisquer em coalizão incluiriam nela todos aqueles que estivessem entre eles para amenizar esse problema, o que poderia gerar coalizões maiores do que as MWC\footnote{A intuição do modelo é a seguinte: se o potencial de conflito entre dois partidos aumenta quanto maior for a distância entre eles numa dimensão qualquer, e se o potencial de conflito numa coalizão é a média dos potenciais de conflitos entre todos os pares não repetidos, coalizões com menor variação nas preferências dos seus membros, isto é, coalizões mais homogêneas, diminuiriam esses conflitos.}. No caso das coalizões mínimas, a explicação é mais direta: um partido pode se beneficiar da agenda do governo e, por isso, votar com ele sem precisar ocupar ministério algum. Entre outros, a principal vantagem de fazer isso seria a de se poupar de desgastes com a administração pública e a de poder competir nas eleições como oposição. Antecipando esse comportamento, o partido minoritário encarregado de formar uma coalizão, o \textit{formateur}, poderia estrategicamente manter partidos próximos ideologicamente fora do governo. Tacitamente, contudo, uma coalizão existiria (STR\O{}M, \citeyear{strom1990}; CITAR MAIS).

As direções a que esses e outros modelos \textit{policy-based} levaram foram diversas. Em termos práticos, a principal delas foi a de tornar a distribuição de preferências ideológicas dos partidos um parâmetro essencial nos modelos de formação de coalizões. Outra, decorrente dessa, criar novas questões sobre o funcionamento das coalizões: como é negociado o programa comum da coalizão? Quais são as dimensões ideológicas que devem ser consideradas pelos modelos? Como se evitam desvios do programa? Se todos os partidos pudessem barganhar livremente e tivessem preferências em múltiplas dimensões, esses modelos espaciais de formação de coalizões sofreriam com problemas de instabilidade (LAVER e SHEPSLE, \citeyear{laver1996}, p. 8-9; SHEPSLE, \citeyear{shepsle1986}; \citeyear{strom1994})\footnote{O debate sobre a ciclicidade de preferências é antigo e complexo para ser abordado aqui de forma sucinta. De qualquer forma, o principal é que ele estava relacionado à impossibilidade de achar um equilíbrio em modelos espaciais multidimensionais: sempre haveria uma coalizão capaz de derrotar o \textit{status quo} e, para que isso fosse evitado, deveria existir alguma regra, como o poder de controlar a agenda de votações, que estruturasse o conjunto das coalizões possíveis (SHEPSLE, \citeyear{shepsle1986}).}. Em última instância, esse problema estimulou o crescimento de uma agenda concorrente, a dos modelos \textit{institution-induced}. Em particular, dois deles merecem atenção pela influência que tiveram na literatura posterior, inclusive na do presidencialismo\footnote{Para uma resenha sobre essa literatura, contendo inclusive um inventário de várias instituições, formais e informais, que poderiam restringir o número de coalizões possíveis e estruturar a formação de coalizões, ver Str\o{}m, Budge e Laver (\citeyear{strom1994}).}.

O primeiro é o de Austen-Smith e Banks (\citeyear{austen1988}). Nele, o \textit{first-move advantage} do \textit{formateur} é modelado, e a chance de um partido vir a integrar uma coalizão é função, além do seu tamanho e de sua posição ideológica, da probabilidade que ele tem de se tornar o próximo \textit{formateur} caso a negociação em curso falhe. Ao contrário de outros autores, portanto, Austen-Smith e Banks analisam a dinâmica do processo de barganha. Seguindo esse \textit{insight}, por exemplo, Crombez (\citeyear{crombez1996}) mostra que formar uma coalizão sobredimensionada pode trazer ganhos de estabilidade. Quanto maior for a probabilidade de que o governo caia por conta da deserção de um de seus membros, maior é o poder de barganha destes, isto é, estes podem usar isso para extrair \textit{side-payments} do \textit{formateur}. Se, por outro lado, este incluir partidos adicionais na coalizão, a única forma do governo cair é se dois ou mais partidos saírem simultaneamente. Entretanto, isso causa um problema coordenativo entre eles: se o partido A, digamos, sai da coalizão e B permanece nela, o governo não cai e os ministérios de A podem ser realocados para B. Ambos, portanto, esperam que o outro saia antes e, no agregado, nada acontece. Carruba e Volden (\citeyear{volden2004}) desenvolvem um modelo com um problema coordenativo semelhante, mas com a utilidade derivada exclusivamente de \textit{policy}\footnote{Nos dois modelos, o problema coordenativo entre os membros da coalizão assemlha-se ao dilema dos prisioneiros, cuja característica principal é a de que desertar é melhor estratégia que cada jogador pode usar independentemente do que os outros jogadores façam.}.

Em suma, o ponto a reter desses modelos de formação de coalizões no parlamentarismo é o seguinte. Partidos buscam formar coalizões para obter suporte legislativo, tanto para maximizar cargos como políticas públicas. Por isso mesmo, número de cadeiras é um bom indicador da necessidade do partido \textit{formateur} buscar ou não uma coalizão; preferências ideológicas, por outro lado, apontam para como os partidos buscam governar e, portanto, são bons indicadores da probabilidade de que dois partidos quaisquer venham a cooperar. Dependendo da posição e do número de cadeiras dos demais, coalizões mínimas ou sobredimensionadas podem surgir. E, por fim, as regras que estruturam o processo de formação de coalizões e exercício do governo devem ser levadas em conta para lidar com variações entre países e contextos e, também, para entender como certas regras podem incentivar o \textit{formateur} a explorar os conflitos entre os membros da coalizão.