\begin{resumo}
Nos últimos anos, a maior parte da literatura sobre as relações executivo-legislativo em sistemas presidencialistas vem enfatizando o papel da formação de coalizões governamentais, através da distribuição de ministérios, na obtenção de maiorias legislativas. Contudo, essas coalizões formadas raramente são iguais, já que umas são maiores e, por causa disso, mais propensas à problemas coordenativos e de agência. Mas o que explica a decisão de um presidente de incluir mais ou menos partidos em seu gabinete? Com um banco de dados original contendo informações sobre 168 coalizões na América Latina entre 1979 e 2012, este artigo testa algumas das hipóteses correntes sobre o fenômeno. Entre outros, os resultados mostram que legislativos fortes e efetivos, presidentes que dispõem de maiores poderes legislativos e maior fragmentação partidária aumentam a probabilidade de ocorrência de coalizões sobredimensionadas em diversas especificações.

\vspace{\onelineskip}
\noindent
\textbf{Palavras-chaves}: Coalizões governamentais;  Presidencialismo; Relações Executivo-Legislativo.
\end{resumo}