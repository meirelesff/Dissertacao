\chapter{Introdução}
\label{chap:introduction}

Quando não contam com maioria no legislativo, presidentes podem incluir outros partidos em seus gabinetes para formar coalizões governamentais e, assim, ampliar suas bases de sustentação. Com a partilha do executivo, os membros do gabinete passam a atuar de forma conjunta para aprovar uma agenda comum, dividir cargos e maximizar o desempenho nas eleições. Este é um dos aspectos centrais enfatizados pela crescente literatura que vê na cooperação interinstitucional a chave para explicar o "inesperado sucesso"~ de sistemas presidencialistas multipartidários (CHAISTY el al., \citeyear{chaisty2014}; MELO e PEREIRA, \citeyear{melo2013}; POWER, \citeyear{power2010}). Contudo, essa explicação deixa ao menos uma maior lacuna. Presidentes muitas vezes formam coalizões com mais partidos do que o necessário para se obter maioria e, como a separação de poderes possibilita a sobrevivência do governo mesmo na ausência de suporte legislativo, as razões que explicam este fenômeno não são claras. Apesar disso, enquanto que as diferentes estratégias que presidentes usam para gerenciar suas coalizões já receberam alguma atenção na literatura (e. g., AMORIM NETO, \citeyear{neto2006}; RAILE et al., \citeyear{raile2010}; ), os fatores que os incentivam a incluir mais ou menos partidos em seus gabinetes, em primeiro lugar, não têm recebido o mesmo tratamento.

Essa negligência é ainda mais problemática quando se procura entender especificamente as consequêncis que coalizões sobredimensionadas, nas quais existem mais partidos do que o necessário para se obter maioria absoluta no congresso, podem trazer. Em primeiro lugar, como Riker (\citeyear{riker1962}) demonstrou, mais membros numa coalizão significa menos cargos à disposição de cada membro do governo e, portanto, formar coalizões sobredimensionadas seria um erro. Em segundo lugar, problemas de coordenação e agência aumentam em coalizões mais fragmentadas. Com um maior número de partidos envolvidos na formulação de uma agenda comum, o número de atores com poder de veto potencialmente aumenta, o conjunto das propostas capazes derrotar o \textit{status quo} diminui e consensos tornam-se mais difíceis de serem mantidos (AXELROD, \citeyear{axelrod1970}). E, em terceiro lugar, torna-se mais difícil fiscalizar as ações de cada membro da coalizão, especialmente quando cada partido usa os seus ministérios para obter vantagens informacionais às custas dos demais (SAALFELD, \citeyear{saalfeld2000}; MARTIN e VANBERG, \citeyear{martin2011})\footnote{Uma leva de estudos argumenta que estes problemas de \textit{moral hazard} decorrentes dessa relação agente-principal são contornáveis (e. g., AMORIM e TAFNER, \citeyear{neto2002}; STR\O{}M et al., \citeyear{strom2010}). Mesmo assim, essas correções trazem custos: esse é ponto principal do argumento aqui utilizado.}. Considerando a frequência com que ocorrem\footnote{Figueiredo et al. (\citeyear{figueiredo2012}, p. 847) reportam que mais de 35\% dos governos na América Latina entre 1979 e 2011 foram supermajoritários}, portanto, cabe perguntar: por que presidentes propõem, e partidos aceitam integrar, coalizões sobredimensionadas? Apesar da importâncias dessa questão, as respostas na literatura são vagas e estudos comparados, inexistentes.
 
Neste artigo, procuro exatamente preencher esta lacuna. Com um banco de dados que cobre todos os 18 países presidencialistas da América Latina após a terceira onda da democratização, testo as principais hipóteses presentes na literatura com análise multivariada. Ao invés de investir na mais recorrente, de que presidentes aumentariam o tamanho de suas coalizões antecipando a indisciplina dos partidos que o apoiam, enfatizo principalmente três outras hipóteses, relacionadas à interação executivo-legislativo: (1) presidentes com maiores poderes legislativos têm menores incentivos para formar coalizões grandes; e (2) congressos fortes e (3) mais fragmentados, ao contrário, incentivam presidentes a formar coalizões sobredimensionadas. A chave para entendê-las está no \textit{policy-making}.

Presidentes que, como os do Brasil e da Argentina, contam com certos poderes legislativos, como o de vetar parcialmente propostas ou o de expedir decretos legislativos, seriam menos propensos à cooperar com o congresso porque poderiam simplesmente impedir ou promover alterações no \textit{status quo} unilateralmente, particularmente quando se deparam com circunstâncias desfavoráveis (NEGRETTO, \citeyear{negretto2006}; PEREIRA et al., \citeyear{pereira2005}). Por outro lado, se o legislativo possuir meios de impedir que o executivo aja dessa forma, o mais provável é que ele consiga obter a concessões -- na maioria das vezes, em termos de partas ministeriais (ALEMAN e TSEBLEIS, \citeyear{aleman2011}; MARTINEZ-GALLARDO, \citeyear{martinez2012}). Especificamente, isto pode ocorrer quando o legislativo produz informação e é capaz de controlar sua própria agenda, quando um país é bicameral e a composição das duas casas difere substancialmente ou quando o congresso possui meios efetivos de fiscalizar o poder executivo. Neste caso, presidentes poderiam incorporar mais partidos em seus gabinetes para reduzir o uso que a oposição pode fazer destas prerrogativas. Esta explicação vai na direção contrária da força dos presidentes e, em conjunto com ela, enfatiza o papel da assimetria de poderes entre executivo e legislativo como uma das chaves para explicar o tamanho das coalizões. Por fim, com mais partidos no congresso o presidente torna-se menos dependente de um partido determinado e, assim, o custo marginal de adicionar um partido extra em seu gabinete pode ser menor do que o de ter de barganhar com a sua coalizão na ausência deste (Cf. VOLDEN e CARRUBA, \citeyear{volden2004}).

Os resultados corroboram duas dessas três hipóteses principais. Congressos capazes de impedir mudanças no \textit{status quo} e de fiscalizar o executivo aumentam a probabilidade de surgirem coalizões sobredimensionadas de forma não desprezível, assim como o aumento da fragmentação partidária. Presidentes que contam com maiores poderes legislativos, por sua vez, ajudam a explicar o fenômeno, mas ao contrário do que boa parte da literatura sustenta, isto é, quanto mais forte o presidente, maior a probabilidade de que ele forme uma coalizão sobredimensionada. O efeito destas variáveis se mantêm em diversos modelos e não são sensíveis nem a mudanças na especificação da variável dependente. Mas por que um presidente que possui as condições para governar de forma unilateral faria justamente o oposto, governando com uma coalizão sobredimensionada? Como sugiro na conclusão, a explicação pode estar relacionada à pouca capacidade explicativa da teoria presidencialista da ação unilateral e à influência que certos poderes legislativos podem ter para o gerenciamento de uma coalizão muito grande.

No resto do artigo, procedo da seguinte forma. Na~\hyperref[chap:revisao]{Seção 2}, discuto brevemente os principais parâmetros nos modelos sobre formação de coalizões governamentais no parlamentarismo e reviso alguns estudos que abordaram variações nos tipos de governos multipartidários no presidencialismo. Na~\hyperref[chap:analise]{Seção 3}, introduzo as hipóteses a serem testadas sobre os determinantes das coalizões sobredimensionadas na Amérila Latina e apresento os métodos e os dados utilizados, que cobrem o período entre 1978 a 2010 e incluem todos os 18 países presidencialistas da América Latinaa. Por fim, na~\hyperref[chap:resultados]{Seção 4} apresento os resultados e os discuto na~\hyperref[chap:discussao]{Seção 5}.
